\documentclass[a4paper,10pt]{article}
\usepackage[utf8x]{inputenc}
\usepackage{amsmath} 
\usepackage{mathtools}
\usepackage{amssymb}
\usepackage{enumitem}
\usepackage{enumerate}

\title{Machine Learning HW5}
\author{Aaron Myers}

\begin{document}

\maketitle
\section{Galois Instllation}
\begin{enumerate}[a)]
  \item 
    \begin{verbatim}
    $ tar xvzf Galois-2.2.1.tar.gz;
    $ cd Galois-2.2.1/build;
    $ mkdir default; cd default;
    $ module load boost/1.55.0
    $ module load cmake
    $ module swap intel gcc/4.7.1 
    $ export BOOST_ROOT=$TACC_BOOST_DIR
    $ cmake ../..
    $ make
    \end{verbatim}
  \item Completed
\end{enumerate}

\section{Sparse Matrix-Vector Multiplication}
\begin{enumerate}[a)]
  \item Completed
  \item Run Times listed below
\begin{center}
	\begin{tabular}{c|c}
	\hline\hline
	Thread & Run Time(s) \\
	\hline \hline
	1 & 0.773 \\
	4 & 0.390 \\
	8 & 0.293 \\
	16 & 0.187
	\end{tabular}
\end{center}
\end{enumerate}

\pagebreak
\section{PageRank}
\begin{enumerate}[a)]
  \item To compute $P^{T}$r, first we need to compute $AD^{-1}$r which will obviously only require O(nnz(A)); since D is diagonal, it will only scale the values in A, creating no new values. Second we compute v$\sum_{i=1}^{n} r_{i}$, which is O(n) and fill in the values of $P^{T}$r where there are zeros in A according to column and add v$\sum_{i=1}^{n} r_{i}$ to non-zero entries of A which is an O(nnz(A)) computation. Therefore the result is O(nnz(A) + n). Below is a derivation for the explanatiion above:
    \begin{align}
      P^{T}r &= ((1-\alpha)D^{-1}A + \alpha1v^{T})^{T}r \\
      &=  (1-\alpha)A^{T}D^{-T}r + \alpha v 1^{T}r \\
      &= (1-\alpha)AD^{-1}r + \alpha v \sum_{i=1}^{n} r_{i}
      \label{}
    \end{align}
  \item Run time: 12.3s, Matlab 45s.
  \item Top 10 Nodes:
\begin{verbatim}
Search for nodes...
Rank: 1, 0.000196396, 4728
Rank: 2, 0.000192605, 2265
Rank: 3, 0.000177875, 238
Rank: 4, 0.000177258, 7587
Rank: 5, 0.000173138, 1506
Rank: 6, 0.000168211, 3655
Rank: 7, 0.000124487, 8556
Rank: 8, 0.000110405, 44092
Rank: 9, 0.000107078, 7100
Rank: 10, 0.000106554, 7556
\end{verbatim}
\end{enumerate}
\end{document}
